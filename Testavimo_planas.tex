\documentclass[12pt]{article}
\usepackage{indentfirst}
\usepackage[utf8x]{inputenc}
\usepackage[T1]{fontenc}
\usepackage[english,lithuanian]{babel}
\usepackage{array}
\usepackage{caption}
\usepackage{subcaption}
\usepackage{makecell}
\usepackage[euler]{textgreek}
\usepackage{multirow}
\usepackage{boldline}
\usepackage{floatrow}
\floatsetup[table]{capposition=top}
\usepackage{amsmath, amsthm, amssymb}
\usepackage{graphicx}
\usepackage{setspace}
\usepackage{verbatim}
\usepackage[left=3cm,top=2cm,right=1.5cm,bottom=2cm]{geometry}
\usepackage{floatrow}
\newfloatcommand{capbtabbox}{table}[][\FBwidth]
\usepackage{blindtext}
\onehalfspacing
\usepackage[hidelinks, unicode]{hyperref}
\usepackage{textcomp}
\usepackage{amsmath}
\usepackage{cleveref}
\usepackage[labelfont=bf]{caption}
\usepackage{microtype}
\usepackage{tabularx}
\captionsetup[table]{font={normalfont},format=plain,labelsep=period}
\graphicspath{{./Assets/}}
\usepackage[table]{xcolor}
\definecolor{newGray}{rgb}{0.878, 0.878, 0.878}
\usepackage{longtable}
\usepackage{enumitem}
\definecolor{deepchampagne}{rgb}{0.98, 0.84, 0.65}
\definecolor{dartmouthgreen}{rgb}{0.09, 0.45, 0.27}
\definecolor{deepcarmine}{rgb}{0.66, 0.13, 0.24}
\usepackage{makecell}

\newcommand{\EE}{\mathbb{E}\,}
\newcommand{\ee}{{\mathrm e}}
\newcommand{\dd}{{\mathrm d}}
\newcommand{\RR}{\mathbb{R}}

\begin{document}
\selectlanguage{lithuanian}

\begin{titlepage}
\vskip 20pt
\begin{center}
\includegraphics[scale=1.4]{KTU.png}
\end{center}

%%%%%%%%%%%%%%%%%%%%%%%
% TITULINIS PUSLAPIS
%%%%%%%%%%%%%%%%%%%%%%%

\vskip 20pt
\centerline{\bf \large \textbf{Kauno technologijos universitetas}}
\bigskip
\centerline{\large {Informatikos fakultetas}}
\bigskip

\vskip 90pt
\begin{center}
    {\bf \LARGE Modulis „Programų kokybės užtikrinimo metodai“}
    \vskip 15pt
    {\large Testavimo planas}
\end{center}

\vskip 40pt

\hskip 200pt {\bf \large IFM 4/2 gr. Danielė Stasiūnaitė}
\vskip 1pt
\hskip 200pt {\large Studentė}
\vskip 7pt
\hskip 200pt {\bf \large Dr. Šarūnas Packevičius}
\vskip 1pt
\hskip 200pt {\large Dėstytojas}

\bigskip

\vskip 100pt
\centerline{\large \textbf{Kaunas, 2025}}
\newpage
\end{titlepage}

\selectlanguage{lithuanian}

%%%%%%%%%%%%%%%%%%%%%
% TURINIO PUSLAPIS
%%%%%%%%%%%%%%%%%%%%%

\tableofcontents
\newpage

%%%%%%%%%%%%%%%%%%%%%%%%%%
% ĮVADAS
%%%%%%%%%%%%%%%%%%%%%%%%%%

\section{Įvadas}
Šiame dokumente yra aprašytas testavimo planas, kuris yra skirtas patikrinti, ar
savarankiškos suverenios asmens tapatybės valdymo sistema, skirta darbui su
asmenų biologiniais duomenimis, veikia taip, kaip yra numatyta parengtoje
sistemos specifikacijoje.

%%%%%%%%%%%%%%%%%%%%%%%%%%
% TESTAVIMO APIMTIS
%%%%%%%%%%%%%%%%%%%%%%%%%%

\section{Testavimo apimtis}
\label{sec:scope}
Numatyta, kad sistemos testavimas bus vykdomas, atliekant visų sistemos
funkcinių dalių testavimą, kur funkcinės dalys yra:
\begin{itemize}
    \item Registracijos modulio veikimas (paskyros kūrimas skirtingoms sistemos
    naudotojų grupėms: pacientams, gydytojams - genetikams, tyrėjams);
    \item Sistemos naudotojų - pacientų - biologinių duomenų prieigos valdymas;
    \item Duomenų šifravimo ir pseudonimizavimo funkcijos;
    \item Biologinių duomenų įkėlimo funkcionalumas;
    \item Biologinių duomenų analizės funkcionalumas;
    \item Užklausų biologinės analizės atlikimui generavimas;
    \item Analizės rezultatų pateiktis.
\end{itemize}

\newpage

%%%%%%%%%%%%%%%%%%%%%%%%%%
% TESTAVIMO STRATEGIJOS
%%%%%%%%%%%%%%%%%%%%%%%%%%

\section{Testavimo strategijos}
Antrame skyriuje (\hyperref[sec:scope]{žr. \emph{antrą skyrių}}) aprašytos
sistemos funkcinės dalys bus testuojamos, taikant šias testavimo strategijas:
\begin{itemize}
    \item \textbf{Vienetų testavimą (angl. \emph{Unit testing})}, kai bus
    testuojami atskiri sistemos komponentai;
    \item \textbf{Integracijos testavimą (angl. \emph{Integration testing})},
    kai bus testuojama, kaip atskiri sistemos komponentai veikia tarpusavyje;
    \item \textbf{Sistemos testavimą (angl. \emph{System testing})}, kai bus
    testuojama, kaip sistema veikia kaip visuma;
    \item \textbf{Priėmimo testavimą (angl. \emph{Acceptance testing})}, kai bus
    tikrinama, ar sistema atitinka visus užsakovo reikalavimus;
    \item \textbf{\emph{Alfa} ir \emph{beta} testavimą}, kai sistemos
    funkcionalumą testuos sistemos kūrėjai (\emph{alfa} testavmas) ir grupelė
    galutinių sistemos naudotojų (\emph{beta} testavimas);
    \item \textbf{Našumo testavimą}, kai bus tikrinama, ar sistema tenkina
    sistemos našumo reikalavimus (veikia pakankamai greitai ir efektyviai);
    \item \textbf{Saugumo testavimą}, kai bus tikrinama, ar sistema yra
    pasiekiama tik autorizuotiems sistemos naudotojams.
\end{itemize}

\newpage

%%%%%%%%%%%%%%%%%%%%%%%%%%
% PRADINĖS SĄLYGOS
%%%%%%%%%%%%%%%%%%%%%%%%%%

\section{Pradinės sąlygos}
Tam, jog galėtų būti pradėtas testavimo etapas, turi būti:

\begin{itemize}
    \item Pilnai suformuota sistemos testavimo komanda, kur kiekvienas komandos
    narys tiksliai žino savo roles ir atsakomybes, atliekant testavimą.
    \item Pilnai paruošta ir patvirtinta sistemos reikalavimų specifikacija,
    pagal kurią būtų galima testuoti sistemos funkcionalumą.
    \item Paruošti ir patvirtinti testavimo scenarijai, pagal kuriuos bus
    atliekamas testavimas.
    \item Paruošti testavimo duomenys, kurie bus naudojami testavimo metu
    (skirtingų kategorijų sistemos naudotojų paskyros, skirtingi biologinių
    duomenų failai, skirtingi analizės atlikimo užklausų pavyzdžiai).
    \item Pilnai paruošta ir sukonfigūruota testavimo aplinka, kurioje bus
    atliekamas sistemos testavimas (duomenų bazės paruošimas, reikalingų
    programinės įrangos komponentų ir testavimo įrankių įdiegimas ir
    konfigūravimas).
    \item Paruoštas testavimo planas, kuriame pateikta visa su testavimu
    susijusi informacija bei pateikti testavimo scenarijai, kuriuose turės būti
    fiksuojamas testavimo rezultatas.
    \item Paruoštas testavimo klaidų ataskaitos šablonas, pagal kurį bus
    fiksuojamos testavimo metu identifikuotos klaidos.
\end{itemize}

\newpage

%%%%%%%%%%%%%%%%%%%%%%%%%%
% TESTAVIMO PRIORITETAI
%%%%%%%%%%%%%%%%%%%%%%%%%%

\section{Testavimo prioritetai}
Testavimas turi būti atliekamas pagal šiuos nustatytus prioritetus:

\begin{itemize}
    \item \textbf{Iš pradžių turi būti testuojami sistemos naudotojui matomi
    sistemos funkciniai reikalavimai}:
    \begin{itemize}[label=$\circ$]
        \item Ar naujas sistemos naudotojas gali užsiregistruoti sistemoje kaip
        pacientas, gydytojas - genetikas arba tyrėjas.
        \item Ar prisijungęs naudotojas pagal savo kategoriją gali pasiekti jam
        skirtą funkcionalumą. Pavyzdžiui, ar pacientas gali įkelti savo
        biologinius duomenis ir valdyti, kokie asmenys gali šiuos duomenis
        pasiekti.
        \item Ar gydytojas - genetikas gali sukurti pacientų medicininės
        kortelės įrašus, pasiekti pacientų biologinius duomenis ir sugeneruoti
        analizės atlikimo užklausą tyrėjui.
        \item Ar tyrėjas gali pasiekti atitinkamų pacientų biologinius duomenis
        (jei iš gydytojo -  genetiko yra gavęs analizės atlikimo užklausą), ar
        gali atlikti biologinių duomenų analizę ir perduoti analizės atlikimo
        rezultatus gydytojui - genetikui.
    \end{itemize}
    \item \textbf{Toliau turi būti testuojami sistemos naudotojui nematomi
    sistemos funkciniai reikalavimai}:
    \begin{itemize}[label=$\circ$]
        \item Ar korektiškai veikia naudotojo autentifikavimo funkcijos.
        \item Ar sistemos naudotojų - pacientų - biologiniai duomenys yra
        korektiškai šifruojami ir pseudonimizuojami.
    \end{itemize}
    \item \textbf{Po funkcinių reikalavimų testavimo vykdomas aukšto prioriteto
    sistemos ne\-funk\-ci\-nių reikalavimų testavimas}:
    \begin{itemize}[label=$\circ$]
        \item Ar autentifikacijos operacijos atlikimo greitis neviršija 1
        sekundės.
        \item Ar paciento biologinių duomenų prieigos patikrinimas netrunka
        ilgiau nei 500 milisekundžių.
    \end{itemize}
    \item \textbf{Galiausiai vykdomas žemo prioriteto sistemos nefunkcinių
    reikalavimų testavimas}:
    \begin{itemize}[label=$\circ$]
        \item Ar naudotojo sąsajos elementai yra tinkamai išdėstyti.
        \item Ar sistemoje realizuoti iššokantys langeliai netrukdo ir neerzina.
        \item Ar sistema yra intuityvi ir pritaikyta vyresnio amžiaus žmonėms.
        \item Ar sistemoje yra pateikti skirtingų techninių terminų bei
        privalomų įvesties laukų paaiškinimai.
    \end{itemize}
\end{itemize}

\newpage

%%%%%%%%%%%%%%%%%%%%%%%%%%
% REZULTATAI
%%%%%%%%%%%%%%%%%%%%%%%%%%

\section{Rezultatai}
Atlikus testavimą bus pilnai užpildyti arba sukurti dokumentai - testavimo
rezultatai:

\begin{itemize}
    \item Bus pilnai užpildytas testavimo planas (prie testavimo scenarijų
    nurodant testavimo rezultatą ir testo būseną);
    \item Bus sukurta testavimo klaidų ataskaita;
    \item Su užsakovu bus pasirašytas dokumentas, patvirtinantis, kad sistema
    atitinka visus užsakovo sistemai iškeltus reikalavimus.
\end{itemize}

\newpage

%%%%%%%%%%%%%%%%%%%%%%%%%%
% TESTAVIMO APLINKA
%%%%%%%%%%%%%%%%%%%%%%%%%%

\section{Testavimo aplinka}
Žemiau aprašyta testavimo aplinkos - aparatinės ir programinės įrangos bei
testavimo įrankių - konfigūracija:

\begin{itemize}
    \item \textbf{Aparatinė įranga}:
    \begin{itemize}[label=$\circ$]
        \item Serveris. 8-16 branduolių CPU, 32-64 GB RAM, SSD diskas (min.
        512 GB).
        \item Darbo kompiuteriai. Testavimui turi būti naudojami kompiuteriai,
        turintys Windows 10/11 ir Linux (Ubuntu 20.04+) operacines sistemas.
    \end{itemize}
    \item \textbf{Programinė įranga}:
    \begin{itemize}[label=$\circ$]
        \item MySQL (8.0+ versija). Duomenų bazės valdymas.
        \item Python (3.10+ versija). \emph{Back-end} API kūrimas ir testavimas.
    \end{itemize}
    \item \textbf{Testavimo įrankiai}:
    \begin{itemize}[label=$\circ$]
        \item PyTest. Vienetų (angl. \emph{unit testing}) ir integracijos
        testavimas.
        \item Selenium. UI testavimas.
        \item Postman. API testavimas.
        \item SoapUI. Saugumo testavimas.
        \item Apache JMeter. Našumo testavimas.
    \end{itemize}
\end{itemize}

\newpage

%%%%%%%%%%%%%%%%%%%%%%%%%%
% TESTAVIMO SCENARIJAI
%%%%%%%%%%%%%%%%%%%%%%%%%%

\section{Testavimo scenarijai}
Žemiau aprašyti keli sistemos funkcionalumo testavimo scenarijai, pagal kuriuos
bus atliekamas testavimas.

\begin{table}[htb!]
    \captionsetup{justification=centering}
    \caption{\small\textbf{Testavimo scenarijus Nr. 1.}}
    \vskip -10pt
    \begin{tabular}{|m{6cm}|m{11cm}|}
        \hline
        \raggedleft \textbf{\cellcolor{deepchampagne}Kodas:} &
        \ttfamily{TS\_001}. \\
        \hline
        \raggedleft \textbf{\cellcolor{deepchampagne}Pavadinimas:} & Naudotojo
        (paciento) paskyros sukūrimo testas. \\
        \hline
        \raggedleft \textbf{\cellcolor{deepchampagne}Tikslas:} &
        Patikrinti, ar sistemos svečias gali susikurti sistemos paskyrą kaip
        sistemos naudotojas - pacientas. \\
        \hline
        \raggedleft \textbf{\cellcolor{deepchampagne}Pradinės testavimo
        sąlygos:} & 
        Sistemos svečias turi būti atsidaręs sistemos paskyros kūrimo langą. \\
        \hline
        \raggedleft \textbf{\cellcolor{deepchampagne}Įvestis:}
        & Svečio vardas, pavardė, el. paštas, telefono numeris, lytis, adresas,
        slaptažodis. \\
        \hline
        \raggedleft \textbf{\cellcolor{deepchampagne}Testo etapai\footnote{Čia
        ir toliau \textcolor{dartmouthgreen}{žalia} spalva pažymėti naudotojo
        veiksmai.}:} & \vskip 5pt
        \makecell[l]{\parbox[t]{11cm}{
            \textbf{1.} \textcolor{dartmouthgreen}{Naudotojas užpildo pateiktos
            asmeninės paskyros kūrimo formos laukus.} \\
            \textbf{2.} \textcolor{dartmouthgreen}{Naudotojas išsaugo įvestą
            informaciją, paspausdamas išsaugojimo mygtuką.} \\
            \textbf{3.} {Parodomas informacinis
            pranešimas, informuojantis apie sėkmingai sukurtą asmeninę
            paskyrą.} \\
            \textbf{4.} {Sistema prijungia naudotoją
            prie jo asmeninės paskyros.} \\
            \textbf{5.} {Sistema atidaro naudotojo
            asmeninės paskyros langą.}
        }} \\
        \hline
        \raggedleft \textbf{\cellcolor{deepchampagne}Tikėtinas rezultatas:}
        & Sukurtas naujas sistemos naudotojas, kuris priklauso kategorijai
        „Pacientas“. \\
        \hline
        \raggedleft \textbf{\cellcolor{deepchampagne}Tikras rezultatas:}
        & \textcolor{red}{\emph{Užpildoma testavimo metu...}} \\
        \hline
        \raggedleft \textbf{\cellcolor{deepchampagne}Būsena:}
        & \textcolor{red}{\emph{Užpildoma testavimo metu: „Testas sėkmingai
        įvykdytas“ arba „Gauta klaida“}}. \\
        \hline
    \end{tabular}
    \label{table:TS_1}
\end{table}

\newpage

\begin{table}[htb!]
    \captionsetup{justification=centering}
    \caption{\small\textbf{Testavimo scenarijus Nr. 2.}}
    \vskip -10pt
    \begin{tabular}{|m{6cm}|m{11cm}|}
        \hline
        \raggedleft \textbf{\cellcolor{deepchampagne}Kodas:} &
        \ttfamily{TS\_002}. \\
        \hline
        \raggedleft \textbf{\cellcolor{deepchampagne}Pavadinimas:} & Biologinių
        duomenų įkėlimo testas. \\
        \hline
        \raggedleft \textbf{\cellcolor{deepchampagne}Tikslas:} &
        Patikrinti, ar sistemos naudotojas - pacientas - gali įkelti savo
        biologinius duomenis į sistemą. \\
        \hline
        \raggedleft \textbf{\cellcolor{deepchampagne}Pradinės testavimo
        sąlygos:} & 
        Sistemos naudotojas - pacientas - turi būti prisijungęs prie sistemos
        ir atsidaręs biologinių duomenų įkėlimo langą. \\
        \hline
        \raggedleft \textbf{\cellcolor{deepchampagne}Įvestis:}
        & Genominių sekų failas su kokybiniais įverčiais (\emph{.fastq}). \\
        \hline
        \raggedleft \textbf{\cellcolor{deepchampagne}Testo etapai:} & \vskip 5pt
        \makecell[l]{\parbox[t]{11cm}{
            \textbf{1.} \textcolor{dartmouthgreen}{Naudotojas užpildo pateiktos
            duomenų įkėlimo formos laukus ir prideda biologinius duomenis
            saugantį failą.} \\
            \textbf{2.} \textcolor{dartmouthgreen}{Naudotojas išsaugo įvestą
            metainformaciją bei pridėtą failą, paspausdamas išsaugojimo
            mygtuką.} \\
            \textbf{3.} Sistema validuoja failo formatą ir turinį. \\
            \textbf{4.} Sistema užšifruoja duomenis ir išsaugo juos duomenų
            bazėje. \\
            \textbf{5.} Sistema priskiria įrašui
            identifikatorių ir susieja jį su naudotojo paskyra. \\
            \textbf{6.} Parodomas informacinis
            pranešimas, informuojantis apie sėkmingai įkeltus duomenis. \\
            \textbf{7.} \textcolor{dartmouthgreen}{Naudotojas peržiūri įkeltų
            duomenų įrašą savo paskyros skiltyje.}
        }} \\
        \hline
        \raggedleft \textbf{\cellcolor{deepchampagne}Tikėtinas rezultatas:}
        & Failas sėkmingai įkeltas, užšifruotas ir išsaugotas sistemoje. \\
        \hline
        \raggedleft \textbf{\cellcolor{deepchampagne}Tikras rezultatas:}
        & \textcolor{red}{\emph{Užpildoma testavimo metu...}} \\
        \hline
        \raggedleft \textbf{\cellcolor{deepchampagne}Būsena:}
        & \textcolor{red}{\emph{Užpildoma testavimo metu: „Testas sėkmingai
        įvykdytas“ arba „Gauta klaida“}}. \\
        \hline
    \end{tabular}
    \label{table:TS_2}
\end{table}

\newpage

\begin{table}[htb!]
    \captionsetup{justification=centering}
    \caption{\small\textbf{Testavimo scenarijus Nr. 3.}}
    \vskip -10pt
    \begin{tabular}{|m{6cm}|m{11cm}|}
        \hline
        \raggedleft \textbf{\cellcolor{deepchampagne}Kodas:} &
        \ttfamily{TS\_003}. \\
        \hline
        \raggedleft \textbf{\cellcolor{deepchampagne}Pavadinimas:} & Biologinių
        duomenų prieigos valdymo testas. \\
        \hline
        \raggedleft \textbf{\cellcolor{deepchampagne}Tikslas:} & Patikrinti, ar
        sistemos naudotojas - pacientas - gali valdyti, kas gali pasiekti jo
        įkeltus biologinius duomenis. \\
        \hline
        \raggedleft \textbf{\cellcolor{deepchampagne}Pradinės testavimo
        sąlygos:} & Sistemos naudotojas - pacientas - turi būti prisijungęs prie
        sistemos ir atsidaręs biologinių duomenų prieigos valdymo langą. \\
        \hline
        \raggedleft \textbf{\cellcolor{deepchampagne}Įvestis:} & Duomenų
        prieigos valdymo lange iš teisių sąrašo reikia pasirinkti konkretiems
        asmenims priskiriamas teises. \\
        \hline
        \raggedleft \textbf{\cellcolor{deepchampagne}Testo etapai:} & \vskip 5pt
        \makecell[l]{\parbox[t]{11cm}{
            \textbf{1.} {Sistema pateikia paciento
            įkeltų biologinių duomenų sąrašą.} \\
            \textbf{2.} \textcolor{dartmouthgreen}{Naudotojas pasirenka
            konkretų biologinių duomenų sąrašo įrašą.} \\
            \textbf{3.} {Sistema pateikia naudotojų,
            turinčių prieigą prie konkrečių biologinių duomenų, sąrašą.} \\
            \textbf{4.} \textcolor{dartmouthgreen}{Naudotojas redaguoja
            suteiktas prieigos teises sistemos naudotojams: pratęsia prieigos
            laikotarpį arba atšaukia prieigą.} \\
            \textbf{5.} \textcolor{dartmouthgreen}{Naudotojas suteikia
            naujas prieigas naujiems sistemos naudotojams.} \\
            \textbf{6.} {Sistema atnaujina naudotojams
            suteiktų prieigų sąrašą.} \\
            \textbf{7.} {Parodomas informacinis
            pranešimas, informuojantis apie sėkmingai atliktą atnaujinimą.} \\
            \textbf{8.} {Sistema informuoja atitinkamus
            sistemos naudotojus apie prieigos teisių pasikeitimus.}
        }} \\
        \hline
        \raggedleft \textbf{\cellcolor{deepchampagne}Tikėtinas rezultatas:}
        & Sistema leidžia sistemos naudotojui - pacientui - valdyti prieigą prie
        duomenų. Asmuo, kuriam suteikiama prieiga prie paciento įkeltų
        biologinių duomenų, turi galėti vykdyti šių duomenų peržiūrą. \\
        \hline
        \raggedleft \textbf{\cellcolor{deepchampagne}Tikras rezultatas:}
        & \textcolor{red}{\emph{Užpildoma testavimo metu...}} \\
        \hline
        \raggedleft \textbf{\cellcolor{deepchampagne}Būsena:}
        & \textcolor{red}{\emph{Užpildoma testavimo metu: „Testas sėkmingai
        įvykdytas“ arba „Gauta klaida“}}. \\
        \hline
    \end{tabular}
    \label{table:TS_3}
\end{table}

\newpage

%%%%%%%%%%%%%%%%%%%%%%%%%%
% TESTAVIMO KALENDORIUS
%%%%%%%%%%%%%%%%%%%%%%%%%%

\section{Testavimo kalendorius}
Žemiau pateiktoje lentelėje yra pateiktas testavimo kalendorius, kuriame
nurodyta skirtingų testavimo etapų pradžia ir pabaiga.

\begin{table}[htb!]
    \captionsetup{justification=centering}
    \caption{\small\textbf{Testavimo kalendorius.}}
    \vskip -10pt
    \begin{tabular}{
        |>{\centering\arraybackslash}m{5cm}
        |>{\centering\arraybackslash}m{4cm}
        |>{\centering\arraybackslash}m{4cm}|
    }
        \hline
        \textbf{\cellcolor{deepchampagne}Testavimo metodas} &
        \textbf{\cellcolor{deepchampagne}Pradžia} &
        \textbf{\cellcolor{deepchampagne}Pabaiga}  \\
        \hline
        \multicolumn{1}{|>{\raggedright\arraybackslash}m{5cm}|}
            {Vienetų testavimas} &
        \multicolumn{1}{>{\raggedright\arraybackslash}m{4cm}|}{2025-09-01} &
        \multicolumn{1}{>{\raggedright\arraybackslash}m{4cm}|}{2025-09-14}\\
        \hline
        \multicolumn{1}{|>{\raggedright\arraybackslash}m{5cm}|}
            {Integracijos testavimas} &
        \multicolumn{1}{>{\raggedright\arraybackslash}m{4cm}|}{2025-09-15} &
        \multicolumn{1}{>{\raggedright\arraybackslash}m{4cm}|}{2025-09-28}\\
        \hline
        \multicolumn{1}{|>{\raggedright\arraybackslash}m{5cm}|}
            {Našumo testavimas} &
        \multicolumn{1}{>{\raggedright\arraybackslash}m{4cm}|}{2025-09-29} &
        \multicolumn{1}{>{\raggedright\arraybackslash}m{4cm}|}{2025-10-12}\\
        \hline
        \multicolumn{1}{|>{\raggedright\arraybackslash}m{5cm}|}
            {Saugumo testavimas} &
        \multicolumn{1}{>{\raggedright\arraybackslash}m{4cm}|}{2025-10-13} &
        \multicolumn{1}{>{\raggedright\arraybackslash}m{4cm}|}{2025-10-26}\\
        \hline
        \multicolumn{1}{|>{\raggedright\arraybackslash}m{5cm}|}
            {Sistemos testavimas} &
        \multicolumn{1}{>{\raggedright\arraybackslash}m{4cm}|}{2025-10-27} &
        \multicolumn{1}{>{\raggedright\arraybackslash}m{4cm}|}{2025-11-16}\\
        \hline
        \multicolumn{1}{|>{\raggedright\arraybackslash}m{5cm}|}
            {\emph{Alfa} ir \emph{beta} testavimas} &
        \multicolumn{1}{>{\raggedright\arraybackslash}m{4cm}|}{2025-11-17} &
        \multicolumn{1}{>{\raggedright\arraybackslash}m{4cm}|}{2025-11-30}\\
        \hline
        \multicolumn{1}{|>{\raggedright\arraybackslash}m{5cm}|}
            {Priėmimo testavimas} &
        \multicolumn{1}{>{\raggedright\arraybackslash}m{4cm}|}{2025-12-01} &
        \multicolumn{1}{>{\raggedright\arraybackslash}m{4cm}|}{2025-12-14}\\
        \hline
    \end{tabular}
    \label{table:TESTAVIMO_KALENDORIUS}
\end{table}

\newpage

\section{Testavimo rizikos}
Žemiau pateiktoje lentelėje yra aprašytos rizikos, kurios gali sutrikdyti
testavimo proceso vykdymą.

\begin{table}[htb!]
    \captionsetup{justification=centering}
    \caption{\small\textbf{Testavimo rizikų aprašymas.}}
    \vskip -10pt
    \begin{tabular}{
        |>{\centering\arraybackslash}m{4cm}
        |>{\centering\arraybackslash}m{6cm}
        |>{\centering\arraybackslash}m{7cm}|
    }
        \hline
        \textbf{\cellcolor{deepchampagne}Rizika} &
        \textbf{\cellcolor{deepchampagne}Aprašymas} &
        \textbf{\cellcolor{deepchampagne}Rizikos tikimybės mažinimas}  \\
        \hline
        \multicolumn{1}{|>{\raggedright\arraybackslash}m{4cm}|}
            {Nepakankamas testuotojų skaičius.} &
        \multicolumn{1}{>{\raggedright\arraybackslash}m{6cm}|}{Testuotojai gali
        dirbti prie kelių skirtingų projektų. Dėl to išauga vėluojančių
        testavimo veiklų rizika.} &
        \multicolumn{1}{>{\raggedright\arraybackslash}m{7cm}|}{Sprinto planavimo
        susitikimų metu suplanuoti visas testavimo veiklas ir, esant poreikiui,
        į projektą įtraukti daugiau žmonių.} \\
        \hline
        \multicolumn{1}{|>{\raggedright\arraybackslash}m{4cm}|}
            {Identifikuojamos kritinės klaidos.} &
        \multicolumn{1}{>{\raggedright\arraybackslash}m{6cm}|}{Testavimo metu
        identifikuojamos klaidos, dėl kurių negalima testuoti kitų sistemos
        funkcionalumų.} &
        \multicolumn{1}{>{\raggedright\arraybackslash}m{7cm}|}{Užtikrinti, kad
        parašytas programinis kodas yra peržiūrimas visų komandai priklausančių
        programuotojų.} \\
        \hline
        \multicolumn{1}{|>{\raggedright\arraybackslash}m{4cm}|}
            {Sistemos realizacijos metu pasikeičia reikalavimai.} &
        \multicolumn{1}{>{\raggedright\arraybackslash}m{6cm}|}{Testavimo metu
        užsakovas išreiškia poreikį pakeisti tam tikrus sistemai keltus
        reikalavimus - testuojamas funkcionalumas gali neatitikti užsakovo
        poreikių.} &
        \multicolumn{1}{>{\raggedright\arraybackslash}m{7cm}|}{Kuriamos
        sistemos reikalavimų analizės metu detaliai išsiaiškinti sistemai
        keliamus funkcinius ir nefunkcinius reikalavimus bei juos pasitvirtinti
        su užsakovu. Su užsakovu pasirašomoje sutartyje nurodyti, kad
        reikšmingai sistemos veikimą keičiančių reikalavimų realizacija galima
        tada ir tik tada, kai pirmoji sistemos versija yra sudiegta į užsakovo
        aplinkas.} \\
        \hline
        \multicolumn{1}{|>{\raggedright\arraybackslash}m{4cm}|}
            {Nestabili arba nepilna testavimo aplinka.} &
        \multicolumn{1}{>{\raggedright\arraybackslash}m{6cm}|}{Testavimo aplinka
        gali būti nepilnai paruošta testavimo vykdymui (duomenų bazės struktūros
        neatitikimai, neteisingai sukonfigūruotas serveris, nepasiekiamas
        API).} &
        \multicolumn{1}{>{\raggedright\arraybackslash}m{7cm}|}{Į sistemos
        kūrimo komandą turi būti įtraukti patyrę programuotojai, analitikai
        bei testuotojai.} \\
        \hline
    \end{tabular}
    \label{table:RIZIKOS}
\end{table}

% KABUTĖS: „ “
\end{document}
