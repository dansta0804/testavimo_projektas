\documentclass[12pt]{article}
\usepackage{indentfirst}
\usepackage[utf8x]{inputenc}
\usepackage[T1]{fontenc}
\usepackage[english,lithuanian]{babel}
\usepackage{array}
\usepackage{caption}
\usepackage{subcaption}
\usepackage{makecell}
\usepackage[euler]{textgreek}
\usepackage{multirow}
\usepackage{boldline}
\usepackage{floatrow}
\floatsetup[table]{capposition=top}
\usepackage{amsmath, amsthm, amssymb}
\usepackage{graphicx}
\usepackage{setspace}
\usepackage{verbatim}
\usepackage[left=3cm,top=2cm,right=1.5cm,bottom=2cm]{geometry}
\usepackage{floatrow}
\newfloatcommand{capbtabbox}{table}[][\FBwidth]
\usepackage{blindtext}
\onehalfspacing
\usepackage[hidelinks, unicode]{hyperref}
\usepackage{textcomp}
\usepackage{amsmath}
\usepackage{cleveref}
\usepackage[labelfont=bf]{caption}
\usepackage{microtype}
\usepackage{tabularx}
\captionsetup[table]{font={normalfont},format=plain,labelsep=period}
\graphicspath{{./assets/}}
\usepackage[table]{xcolor}
\definecolor{newGray}{rgb}{0.878, 0.878, 0.878}
\usepackage{longtable}
\usepackage{enumitem}
\definecolor{deepchampagne}{rgb}{0.98, 0.84, 0.65}
\definecolor{dartmouthgreen}{rgb}{0.09, 0.45, 0.27}
\definecolor{deepcarmine}{rgb}{0.66, 0.13, 0.24}
\usepackage{makecell}

\newcommand{\EE}{\mathbb{E}\,}
\newcommand{\ee}{{\mathrm e}}
\newcommand{\dd}{{\mathrm d}}
\newcommand{\RR}{\mathbb{R}}

\begin{document}
\selectlanguage{lithuanian}

\begin{titlepage}
\vskip 20pt
\begin{center}
\includegraphics[scale=1.4]{KTU.png}
\end{center}

%%%%%%%%%%%%%%%%%%%%%%%
% TITULINIS PUSLAPIS
%%%%%%%%%%%%%%%%%%%%%%%

\vskip 20pt
\centerline{\bf \large \textbf{Kauno technologijos universitetas}}
\bigskip
\centerline{\large {Informatikos fakultetas}}
\bigskip

\vskip 90pt
\begin{center}
    {\bf \LARGE Modulis „Programų kokybės užtikrinimo metodai“}
    \vskip 15pt
    {\large Testavimo planas}
\end{center}

\vskip 40pt

\hskip 200pt {\bf \large IFM 4/2 gr. Danielė Stasiūnaitė}
\vskip 1pt
\hskip 200pt {\large Studentė}
\vskip 7pt
\hskip 200pt {\bf \large Dr. Šarūnas Packevičius}
\vskip 1pt
\hskip 200pt {\large Dėstytojas}

\bigskip

\vskip 100pt
\centerline{\large \textbf{Kaunas, 2025}}
\newpage
\end{titlepage}

\selectlanguage{lithuanian}

%%%%%%%%%%%%%%%%%%%%%
% TURINIO PUSLAPIS
%%%%%%%%%%%%%%%%%%%%%

\tableofcontents
\newpage

%%%%%%%%%%%%%%%%%%%%%
% ĮVADAS
%%%%%%%%%%%%%%%%%%%%%

\section{Įvadas}
Šiame dokumente yra aprašytas testavimo planas, kuris yra skirtas patikrinti, ar
savarankiškos suverenios asmens tapatybės valdymo sistema, skirta darbui su
asmenų biologiniais duomenimis, veikia taip, kaip yra sistemos speifikacijoje.

\section{Testavimo apimtis}
Bus vykdomas šių sistemos dedamųjų dalių testavimas:
\begin{itemize}
    \item Registracijos modulio veikimas (paskyros kūrimas skirtingoms sistemos
    naudotojų grupėms: pacientams, gydytojams - genetikams, tyrėjams);
    \item Sistemos naudotojų - pacientų - biologinių duomenų prieigos valdymas;
    \item Duomenų šfravimo ir pseudonimizavimo funkcijos;
    \item Biologinių duomenų įkėlimo funkcionalumas;
    \item Biologinių duomenų analizės funkcionalumas;
    \item Užklausų biologinės analizės atlikimui generavimas;
    \item Analizės rezultatų pateiktis.
\end{itemize}

\newpage

\section{Testavimo strategijos}
Aukščiau aprašyti sistemos funkcionalumai bus testuojami, taikant šias testavimo
strategijas:
\begin{itemize}
    \item \textbf{Vienetų testavimą (angl. \emph{Unit testing})}, kai bus
    testuojami atskiri sistemos komponentai;
    \item \textbf{Integracijos testavimą (angl. \emph{Integration testing})},
    kai bus testuojama, kaip atskiri sistemos komponentai veikia tarpusavyje;
    \item \textbf{Sistemos testavimą (angl. \emph{System testing})}, kai bus
    testuojama, kaip sistema veikia kaip visuma;
    \item \textbf{Priėmimo testavimą (angl. \emph{Acceptance testing})}, kai bus
    tikrinama, ar sistema atitinka visus užsakovo reikalavimus;
    \item \textbf{Alfa ir beta testavimą}, kai sistemos funkcionalumą testuos
    sistemos kūrėjai (alfa testavmas) ir grupelė galutinių sistemos naudotojų
    (beta testavimas);
    \item \textbf{Našumo testavimą}, kai bus tikrinama, ar sistema tenkina
    sistemos našumo reikalavimuas (veikia pakankamai greitai ir efektyviai);
    \item \textbf{Saugumo testavimą}, kai bus tikrinama, ar sistema yra
    pasiekiama tik autorizuotiems sistemos naudotojams.
\end{itemize}

\newpage

\section{Pradinės sąlygos}
Tam, jog galėtų būti pradėtas testavimo etapas, turi būti įvykdytos šios
pradinės sąlygos:

\begin{itemize}
    \item Pilnai paruošta ir patvirtinta sistemos reikalavimų specifikacija,
    pagal kurią būtų galima testuoti sistemos funkcionalumą;
    \item Paruošti testavimo scenarijai, pagal kuriuos bus atliekamas
    testavimas;
    \item Paruošti testavimo duomenys, kurie bus naudojami testavimo metu
    (skirtingų kategorijų sistemos naudotojų paskyros, skirtingi biologinių
    duomenų failai, skirtingi analizės atlikimo užklausų pavyzdžiai);
    \item Paruoštas testavimo planas, kuriame pateikta visa su testavimu
    susijusi informacija bei pateikti testavimo scenarijai, kuriuose turės būti
    fiksuojamas testavimo rezultatas.
    \item Paruoštas testavimo klaidų ataskaitos šablonas, pagal kurį bus
    fiksuojamos testavimo metu rastos klaidos.
\end{itemize}

\newpage

\section{Testavimo prioritetai}
Testavimas turi būti atliekamas pagal šiuos nustatytus prioritetus:

\begin{itemize}
    \item \textbf{Iš pradžių turi būti testuojami sistemos naudotojui matomi
    sistemos funkciniai reikalavimai}. T.y., ar naujas sistemos naudotojas gali
    užsiregistruoti sistemos kaip pacientas, gydytojas - genetikas arba tyrėjas;
    ar prisijungęs naudotojas pagal savo kategoriją gali pasiekti jam skirtą
    funkcionalumą. Pavyzdžiui, ar pacientas gali įkelti savo biologinius
    duomenis ir valdyti, kokie asmenys gali šiuos duomenis pasiekti; ar
    gydytojas - genetikas gali sukurti pacientų medicininės kortelės įrašus,
    pasiekti pacientų biologinius duomenis ir sugeneruoti analizės atlikimo
    užklausą tyrėjui; ar tyrėjas gali pasiekti atitinkamų pacientų biologinius
    duomenis (jei iš gydytojo -  genetiko yra gavęs analizės atlikimo užklausą),
    ar gali atlikti biologinių duomenų analizę ir perduoti analizės atlikimo
    rezultatus gydytojui - genetikui.
    \item \textbf{Toliau turi būti testuojami sistemos naudotojui nematomi
    sistemos funkciniai reikalavimai}. T.y., ar korektiškai veikia naudotojo
    autentifikavimo funkcijos ir ar sistemos naudotojų - pacientų - biologiniai
    duomenys yra korektiškai šifruojami ir pseudonimizuojami.
    \item \textbf{Vykdomas aukšto prioriteto sistemos nefunkcinių reikalavimų
    testavimas}. T.y., testuojama, ar autentifikacijos operacijos atlikimo
    greitis neviršija 1 sekundės, ar paciento biologinių duomenų prieigos
    patikrinimas netrunka ilgiau nei 500 milisekundžių.
    \item \textbf{Galiausiai vykdomas žemo prioriteto sistemos nefunkcinių
    reikalavimų testavimas}. T.y., ar naudotojo sąsajos elementai yra tinkamai
    išdėstyti, ar sistemoje realizuoti iššokantys langeliai netrukdo ir
    neerzina, ar sistema yra intuityvi ir pritaikyta vyresnio amžiaus žmonėms,
    ar sistemoje yra pateikti skirtingų techninių terminų bei reikalaujamų
    įvesties laukų paaiškinimai.
\end{itemize}

\newpage

\section{Rezultatai}
Atlikus testavimą bus pateikti šie rezultatai:
\begin{itemize}
    \item Testavimo planas;
    \item Testavimo klaidų ataskaita;
    \item Užsakovo pasirašytas dokumentas, patvirtinantis, kad sistema atitinka
    visus reikalavimus.
\end{itemize}

\newpage

\section{Testavimo aplinka}
\newpage

\section{Testavimo scenarijai}
Žemiau aprašyti keli sistemos funkcionalumo testavimo scenarijai, pagal kuriuos
bus atliekamas testavimas.

\begin{table}[htb!]
    \captionsetup{justification=centering}
    \caption{\small\textbf{Testavimo scenarijus Nr. 1.}}
    \vskip -10pt
    \begin{tabular}{|m{6cm}|m{11cm}|}
        \hline
        \raggedleft \textbf{\cellcolor{deepchampagne}Kodas:} &
        \ttfamily{TS\_001}. \\
        \hline
        \raggedleft \textbf{\cellcolor{deepchampagne}Pavadinimas:} & Naudotojo
        paskyros sukūrimas. \\
        \hline
        \raggedleft \textbf{\cellcolor{deepchampagne}Tikslas:} &
        Patikrinti, ar sistemos svečias gali susikurti sistemos paskyrą kaip
        sistemos naudotojas - pacientas. \\
        \hline
        \raggedleft \textbf{\cellcolor{deepchampagne}Pradinės testavimo
        sąlygos:} & 
        Sistemos svečias turi būti atsidaręs paskyros kūrimo langą. \\
        \hline
        \raggedleft \textbf{\cellcolor{deepchampagne}Įvestis:}
        & Svečio vardas, pavardė, el. paštas, telefono numeris, lytis, adresas,
        slaptažodis. \\
        \hline
        \raggedleft \textbf{\cellcolor{deepchampagne}Tikėtinas rezultatas:}
        & Sukurtas naujas sistemos naudotojas, kuris priklauso kategorijai
        „Pacientas“. \\
        \hline
        \raggedleft \textbf{\cellcolor{deepchampagne}Tikras rezultatas:}
        & \textcolor{red}{\emph{Užpildoma testavimo metu...}} \\
        \hline
        \raggedleft \textbf{\cellcolor{deepchampagne}Būsena:}
        & \textcolor{red}{\emph{Užpildoma testavimo metu: „Testas sėkmingai
        įvykdytas“ arba „Gauta klaida“}}. \\
        \hline
    \end{tabular}
    \label{table:TS_1}
\end{table}

\begin{table}[htb!]
    \captionsetup{justification=centering}
    \caption{\small\textbf{Testavimo scenarijus Nr. 2.}}
    \vskip -10pt
    \begin{tabular}{|m{6cm}|m{11cm}|}
        \hline
        \raggedleft \textbf{\cellcolor{deepchampagne}Kodas:} &
        \ttfamily{TS\_002}. \\
        \hline
        \raggedleft \textbf{\cellcolor{deepchampagne}Pavadinimas:} & Biologinių
        duomenų įkėlimas. \\
        \hline
        \raggedleft \textbf{\cellcolor{deepchampagne}Tikslas:} &
        Patikrinti, ar sistemos naudotojas - pacientas - gali įkelti savo
        biologinius duomenis į sistemą. \\
        \hline
        \raggedleft \textbf{\cellcolor{deepchampagne}Pradinės testavimo
        sąlygos:} & 
        Sistemos naudotojas - pacientas - turi būti prisijungęs prie sistemos
        ir atsidaręs biologinių duomenų įkėlimo langą. \\
        \hline
        \raggedleft \textbf{\cellcolor{deepchampagne}Įvestis:}
        & Genominių sekų failas su kokybiniais įverčiais (\emph{.fastq}). \\
        \hline
        \raggedleft \textbf{\cellcolor{deepchampagne}Tikėtinas rezultatas:}
        & Failas sėkmingai įkeltas, užšifruotas ir išsaugotas sistemoje. \\
        \hline
        \raggedleft \textbf{\cellcolor{deepchampagne}Tikras rezultatas:}
        & \textcolor{red}{\emph{Užpildoma testavimo metu...}} \\
        \hline
        \raggedleft \textbf{\cellcolor{deepchampagne}Būsena:}
        & \textcolor{red}{\emph{Užpildoma testavimo metu: „Testas sėkmingai
        įvykdytas“ arba „Gauta klaida“}}. \\
        \hline
    \end{tabular}
    \label{table:TS_2}
\end{table}

\begin{table}[htb!]
    \captionsetup{justification=centering}
    \caption{\small\textbf{Testavimo scenarijus Nr. 3.}}
    \vskip -10pt
    \begin{tabular}{|m{6cm}|m{11cm}|}
        \hline
        \raggedleft \textbf{\cellcolor{deepchampagne}Kodas:} &
        \ttfamily{TS\_003}. \\
        \hline
        \raggedleft \textbf{\cellcolor{deepchampagne}Pavadinimas:} & . \\
        \hline
        \raggedleft \textbf{\cellcolor{deepchampagne}Tikslas:} &
        . \\
        \hline
        \raggedleft \textbf{\cellcolor{deepchampagne}Pradinės testavimo
        sąlygos:} & 
        . \\
        \hline
        \raggedleft \textbf{\cellcolor{deepchampagne}Įvestis:}
        & . \\
        \hline
        \raggedleft \textbf{\cellcolor{deepchampagne}Tikėtinas rezultatas:}
        & . \\
        \hline
        \raggedleft \textbf{\cellcolor{deepchampagne}Tikras rezultatas:}
        & \textcolor{red}{\emph{Užpildoma testavimo metu...}} \\
        \hline
        \raggedleft \textbf{\cellcolor{deepchampagne}Būsena:}
        & \textcolor{red}{\emph{Užpildoma testavimo metu: „Testas sėkmingai
        įvykdytas“ arba „Gauta klaida“}}. \\
        \hline
    \end{tabular}
    \label{table:TS_3}
\end{table}

\newpage

\section{Testavimo kalendorius}
\newpage

\section{Testavimo rizikos}
Žemiau pateiktoje lentelėje yra aprašytos rizikos, kurios gali sutrikdyti
testavimo proceso vykdymą.

\begin{table}[htb!]
    \captionsetup{justification=centering}
    \caption{\small\textbf{Testavimo rizikų aprašymas.}}
    \vskip -10pt
    \begin{tabular}{
        |>{\centering\arraybackslash}m{4cm}
        |>{\centering\arraybackslash}m{5cm}
        |>{\centering\arraybackslash}m{7cm}|
    }
        \hline
        \textbf{\cellcolor{deepchampagne}Rizika} &
        \textbf{\cellcolor{deepchampagne}Aprašymas} &
        \textbf{\cellcolor{deepchampagne}Rizikos tikimybės mažinimas}  \\
        \hline
        \multicolumn{1}{|>{\raggedright\arraybackslash}m{4cm}|}
            {...} &
        \multicolumn{1}{>{\raggedright\arraybackslash}m{5cm}|}{...} &
        \multicolumn{1}{>{\raggedright\arraybackslash}m{7cm}|}{...}\\
        \hline
    \end{tabular}
    \label{table:RIZIKOS}
\end{table}

% KABUTĖS: „ “
\end{document}
